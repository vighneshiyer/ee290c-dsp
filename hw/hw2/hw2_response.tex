\input{../common.tex}

\newcommand{\headertext}{EE290C HW 2}
\renewcommand{\thesubsection}{\thesection.\alph{subsection}}

\title{\vspace{-0.4in}\Large \bf \headertext \vspace{-0.1in}}
\author{Vighnesh Iyer}

\date{\today}
\maketitle

\markboth{\headertext}{\headertext}
\thispagestyle{empty}

\section{Bitwidth Inference}
{\color{blue} For the 'Example' circuit, use FIRRTL's conservative bitwidth inference rules to derive the widths of registers \verb|sum| and \verb|prod| and output \verb|out|. Check the results using the FIRRTL compiler.}

\begin{align*}
    w_{sum} &= \max(w_{a}, w_{b}) + 1 = 5 \\
    w_{prod} &= w_{sum} + w_{c} = 8 \\
    w_{out} &= \max(w_{prod}, w_{sum}) + 1 = 9
\end{align*}

The generated Verilog from running \verb|firrtl -i hw2.fir -o hw2.v| contains these lines which verify the width calculation.

\begin{minted}{verilog}
module Example( // @[:hw2.fir@2.4]
  input        clk, // @[:hw2.fir@3.8]
  input  [3:0] a, // @[:hw2.fir@4.8]
  input  [1:0] b, // @[:hw2.fir@5.8]
  input  [2:0] c, // @[:hw2.fir@6.8]
  output [8:0] out // @[:hw2.fir@7.8]
);
  reg [4:0] sum; // @[:hw2.fir@9.8]
  reg [7:0] prod; // @[:hw2.fir@10.8]
  ...
endmodule
\end{minted}

\section{Chisel Generator Bootcamp}
The IPython notebooks and HTML are attached for bootcamp sections 2.3-2.5 and 3.1-3.2.

\end{document}
