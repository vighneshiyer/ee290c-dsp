\documentclass[11pt]{article}

% Package includes

\usepackage{graphicx}
\usepackage{color}
\usepackage{comment}
\usepackage{multirow}
\usepackage{askmaps}
\usepackage{amssymb}
\usepackage{amsmath}
\usepackage{tikz}
\usetikzlibrary{arrows, positioning, shapes.geometric, circuits.logic.US}
\tikzstyle{line}=[draw]
\tikzstyle{arrow}=[draw, -latex]

% Wrap long URLs with hyphens
\PassOptionsToPackage{hyphens}{url}\usepackage{hyperref}
\usepackage{pdftexcmds}
\usepackage{upquote}
\usepackage{textcomp}
\usepackage{minted}
\usepackage[listings]{tcolorbox}
\usepackage{enumerate}
\usepackage{enumitem}
\usepackage{mathtools}
\DeclarePairedDelimiter{\ceil}{\Big\lceil}{\Big\rceil}

\tcbset{
texexp/.style={colframe=black, colback=lightgray!15,
         coltitle=white,
         fonttitle=\small\sffamily\bfseries, fontupper=\small, fontlower=\small},
     example/.style 2 args={texexp,
title={Question \thetcbcounter: #1},label={#2}},
}

\newtcolorbox{texexp}[1]{texexp}
\newtcolorbox[auto counter]{texexptitled}[3][]{%
example={#2}{#3},#1}

\setlength{\topmargin}{-0.5in}
\setlength{\textheight}{9in}
\setlength{\oddsidemargin}{0in}
\setlength{\evensidemargin}{0in}
\setlength{\textwidth}{6.5in}

% Useful macros

\newcommand{\note}[1]{{\bf [ NOTE: #1 ]}}
\newcommand{\fixme}[1]{{\bf [ FIXME: #1 ]}}
\newcommand{\wunits}[2]{\mbox{#1\,#2}}
\newcommand{\um}{\mbox{$\mu$m}}
\newcommand{\xum}[1]{\wunits{#1}{\um}}
\newcommand{\by}[2]{\mbox{#1$\times$#2}}
\newcommand{\byby}[3]{\mbox{#1$\times$#2$\times$#3}}

\newenvironment{tightlist}
{\begin{itemize}
 \setlength{\parsep}{0pt}
 \setlength{\itemsep}{-2pt}}
{\end{itemize}}

\newenvironment{titledtightlist}[1]
{\noindent
 ~~\textbf{#1}
 \begin{itemize}
 \setlength{\parsep}{0pt}
 \setlength{\itemsep}{-2pt}}
{\end{itemize}}

% Change spacing before and after section headers

\makeatletter
\renewcommand{\section}
{\@startsection {section}{1}{0pt}
 {-2ex}
 {1ex}
 {\bfseries\Large}}
\makeatother

\makeatletter
\renewcommand{\subsection}
{\@startsection {subsection}{1}{0pt}
 {-1ex}
 {0.5ex}
 {\bfseries\normalsize}}
\makeatother

% Reduce likelihood of a single line at the top/bottom of page

\clubpenalty=2000
\widowpenalty=2000

% Other commands and parameters

\pagestyle{myheadings}
\setlength{\parindent}{0in}
\setlength{\parskip}{10pt}

% Commands for register format figures.

\newcommand{\instbit}[1]{\mbox{\scriptsize #1}}
\newcommand{\instbitrange}[2]{\instbit{#1} \hfill \instbit{#2}}

\graphicspath{{./figs/}}

%-----------------------------------------------------------------------
% Document
%-----------------------------------------------------------------------

\begin{document}
\def\PYZsq{\textquotesingle}


\newcommand{\headertext}{EE290C HW 2}
\renewcommand{\thesubsection}{\thesection.\alph{subsection}}

\title{\vspace{-0.4in}\Large \bf \headertext \vspace{-0.1in}}
\author{Vighnesh Iyer}

\date{\today}
\maketitle

\markboth{\headertext}{\headertext}
\thispagestyle{empty}

\section{Bitwidth Inference}
{\color{blue} For the 'Example' circuit, use FIRRTL's conservative bitwidth inference rules to derive the widths of registers \verb|sum| and \verb|prod| and output \verb|out|. Check the results using the FIRRTL compiler.}

\begin{align*}
    w_{sum} &= \max(w_{a}, w_{b}) + 1 = 5 \\
    w_{prod} &= w_{sum} + w_{c} = 8 \\
    w_{out} &= \max(w_{prod}, w_{sum}) + 1 = 9
\end{align*}

The generated Verilog from running \verb|firrtl -i hw2.fir -o hw2.v| contains these lines which verify the width calculation.

\begin{minted}{verilog}
module Example( // @[:hw2.fir@2.4]
  input        clk, // @[:hw2.fir@3.8]
  input  [3:0] a, // @[:hw2.fir@4.8]
  input  [1:0] b, // @[:hw2.fir@5.8]
  input  [2:0] c, // @[:hw2.fir@6.8]
  output [8:0] out // @[:hw2.fir@7.8]
);
  reg [4:0] sum; // @[:hw2.fir@9.8]
  reg [7:0] prod; // @[:hw2.fir@10.8]
  ...
endmodule
\end{minted}

\section{Chisel Generator Bootcamp}
The IPython notebooks and HTML are attached for bootcamp sections 2.3-2.5 and 3.1-3.2.

\end{document}
